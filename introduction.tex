\chapter{Introduction}\label{chap:intro}

\section{Motivation}
Chemical engineering process modelling and simulation pose a
significant challenge to the computer program developer.  Chemical
processes are invariably described by nonlinear equations such as
chemical reaction kinetics, flow-pressure relationships and physical
properties.  Dynamic simulation of such systems involves the solution of
sets of nonlinear differential and algebraic equations.  There is also
uncertainty associated with the model equations themselves (model
uncertainty), their parameters (parametric uncertainty) and the inputs
into the model (input uncertainty).

This work aims to present a framework for the development of
stochastic dynamic simulations of chemical processes. The framework is
designed to be generic. The large problem of stochastic dynamic
modelling has been broken down into three steps: input modelling,
process modelling and postprocessing.

\section{Structure of the document}
This document is divided into three sections: Part I reviews relevant literature, Part II explains the implementation details of 

Each step of the simulation is
discussed

\section{Contributions}
The deliverables of this work are:
\begin{enumerate}
\item A novel application of multi-objective optimisation to the identification of transition probabilities in Markov Chain models of input sequences.
\item A set of Modelica models that enable modelling of chemical engineering unit operations
\item Support for using CAPE-Open compliant property databases for the calculation of thermodynamic properties in these models
\item A set of programs that can run models built using this framework in Monte Carlo simulations
\item 
\end{enumerate}

% Local Variables:
% TeX-master: "thesis"
% End:
