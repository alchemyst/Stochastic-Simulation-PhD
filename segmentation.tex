\chapter{Segmentation}\label{chap:lit:segmentation}
\begin{overview}
  Segmentation may be viewed as a generalisation of trend fitting with multiple curves.
  This chapter provides an overview of segmentation techniques that are commonly deployed both in time series analysis and other fields.
\end{overview}

Segmentation aims to approximate an input signal of length $N$ by $n<N$ events.
Segmentation can be used as a compression measure, as a method of smoothing the data or to investigate underlying structure in the signal.
\citet{keogh.chu.ea1993segmenting} gives a good review of several segmentation algorithms applied to EKG time series.
Segmentation is also used in a different sense in fields like speech recognition to mean identification of transitions in the data without explicit fitting of a curve or reduction of data.
This usage will not be discussed here.

The most popular event type is straight line or piecewise-linear segmentation.
However, more interesting functions like general polynomials \citep{arora.khot2003fitting} have been proposed.

\section{Objectives}
\label{sec:objectives}

A good segmentation algorithm:
\begin{enumerate}
\item minimises the error of the segmented description (or at least satisfies some upper bound on the error),
\item uses the simplest description possible for the data (which may be in terms of the number or complexity of the identified segments) and
\item is efficient in computer time and space requirements.
\end{enumerate}
If the algorithm is to be used on-line to segment signals as they are read, it is also beneficial if the algorithm works can incorporate new
data efficiently.

Some of these objectives are contradictory -- a more complex description will almost always allow a lower segmentation error than a simpler one, for instance.
Also, it is always possible to segment with zero error by simply dividing the segmented data at every single sample point, so direct minimisation of the fitting error is clearly insufficient on its own.

The next sections summarise commonly employed algorithms.
The details are taken largely from~\citet{keogh.chu.ea1993segmenting}, with some reinterpretation to fit within the structure of this chapter.
Note that the algorithms have been significantly reworked.

\section{Top-down methods}
These methods can also be described as subdivision methods and feature a recursive subdivision of the signal that stops when an error measure has been reduced below a threshold.
\begin{algorithm}
  \caption{Top-down algorithm}
  \label{alg:topdown}
  \begin{algorithmic}
    \Function{topdown}{$T, \epsilon$}
    \If{approximationerror($T$) $< \epsilon$}
      \Return{approximate($T$)}
    \Else
      \State $N \gets $ length($T$)
      \State $b \gets \min_i{\mathrm{splitcost}(T,i)}$ \Comment{Find best split point}
      \Return topdown($T[1\dots b]$) + topdown($T[b+1\dots N]$)
    \EndIf
    \EndFunction
\end{algorithmic}
\end{algorithm}
The algorithm described in Algorithm~\ref{alg:topdown} lends itself to optimisation by dynamic programming, as the optimal subdivisions of smaller sequences can be stored as partial solutions to the larger problem.
The Douglas-Peuker algorithm \citep{douglas.peucker1973algorithms} is also an example of a top-down algorithm, although it does not search for optimal breaks recursively -- it simply uses the node with the maximum perpendicular distance from the line as the break point.

\section{Bottom-up methods}
Bottom-up or composition methods start with segments between all data points and merge similar segments until there are no segment pairs to merge without violating an error measure.

\begin{algorithm}
  \caption{Bottom-up algorithm}
  \label{alg:bottomup}
  \begin{algorithmic}
    \Function{bottomup}{$T,\epsilon$}
    \State $N \gets $ length($N$)
    \For {$i \in (1\dots N-1)$} \Comment{Start with lines between
      all points}
    \State segments.append(segment($T[i\dots i+1]$))
    \EndFor
    \For {$i \in 1 \dots$ length(segments)-1} \Comment{Find cost of merging each pair}
    \State $c(i) \gets $error([merge(segments[$i$],segments[$i+1$])])
    \EndFor
    \While{$\min(c)<\epsilon$}
    \State $i \gets \mathrm{minindex}(c)$ \Comment{Find ``cheapest''
      pair to merge}
    \State segments[$i$] $\gets$ merge(segments[$i$],segments[$i+1$]) \Comment{Merge them}
    \State delete(segments[$i+1$]) \Comment{Update records}
    \State $c(i) \gets $error(merge(segments[$i$],segments[$i+1$]);
    \State $c(i-1) \gets $ error(merge(segments[$i-1$], segments[$i$]));
    \EndWhile
    \EndFunction
  \end{algorithmic}
\end{algorithm}

Algorithm~\ref{alg:bottomup} shows a sample algorithm for a bottom-up
method.

If properly implemented, both bottom-up and top-down methods should give similar results.

\section{Methods employing sliding windows}
Sliding window or incremental methods process the signal to be segmented sequentially, in one pass.
This means that they can be employed on-line, in contrast to the recursive methods discussed before, which require the entire data set to be loaded into memory before being started.

\begin{algorithm}
  \caption{Sliding window algorithm}
  \label{alg:slidingwindow}
  \begin{algorithmic}
    \Function{slidingwindow}{$T, \epsilon$}
    \State $a \gets 1$
    \State segments $\gets\emptyset$
    \While{$b<N$}
    \State $b\gets a+1$
    \While{error(T[a$\dots$b]) $ < \epsilon$}
    \State $b \gets b + 1$
    \EndWhile
    \State segments.append(T[$a\dots b-1$])
    \State $a \gets b + 1$
    \EndWhile
    \Return{segments}
    \EndFunction
  \end{algorithmic}
\end{algorithm}

Algorithm~\ref{alg:slidingwindow} shows a possible sliding window algorithm.

\section{Optimisation-based methods}
Any one of the objectives mentioned in Section~\ref{sec:objectives} can be rewritten as an objective function for an optimisation algorithm.
This objective function could then be minimised by choosing the number of parameters, the number of segments and the parameter values for each of the identified segments.
Application of this reasoning can be seen in the direct fitting of line segments to data described in~\citet{cantoni1971optimal}, which leads to a direct analytical solution via the pseudo-inverse, or in where numerical optimisation is employed to fit more complicated segmentation functions.

A common thread in the optimisation-based methods is that the number of line segments must be known in advance.
This is required when using derivative-based optimisation, as the number of design variables fixes the dimensions of the derivative and the current position in the design space.
This is a disadvantage when compared to the previous methods, which would automatically fit varying numbers of segments given different data set.
Recall, however, that these methods would terminate when a certain error bound had been met, and that this bound had to be set in advance.
When the error is reduced using optimisation, this bound is not required.

Another, more significant, benefit of using optimisation rather than the direct methods, is that it enables a more general description of a segment to be used with very little additional effort beyond deciding on the parameters of the description.
While it is clear how to approach subdivision for line segments, it is not as simple to adjust the algorithms for other functions~\citep{waibel.lee1990readings}

% Local Variables:
% TeX-master: "thesis"
% End:
