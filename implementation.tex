\chapter{Implementation}\label{chap:implementation}
\begin{overview}

\end{overview}

\section{Event identification}
\subsection{Application}
As an example of the results obtainable using multi-objective
optimisation, the MOPSO-CD (Multi-Objective Particle Swarm
Optimisation with Crowding Distance) algorithm proposed
by\cite{raquel_effective_2005}was used to fit a first-order response
prototype to input signals.  The algorithm is a modification of
Particle Swarm Optimisation that adds an archive of nondominated
solutions and uses a crowding distance measure to prevent many similar
Pareto optimal solutions from being retained in the archive.

A problem description based on a prototypical first order response was
used in this study. Figure~\ref{fig:definition} shows the prototype
function.
\begin{figure}[htbp]
  \centering
  \setlength{\unitlength}{1.8em}
  \begin{picture}(10,10) 
    \thicklines
    % axis
    \put(1,1){\vector(1,0){8}}
    
    \put(5,0){$t$}
    \put(1,1){\vector(0,1){8}}
    \put(0,7){$y_p$}
    % curve
    \qbezier(2,2)(4,8)(8,8)
    \put(2,2){\circle*{0.2}}
    \put(2,0){$t_{i-1}$}
    \put(0,2){$y_{i-1}$}

    \put(8,8){\circle*{0.2}} 
    \put(8,0){$t_i$}
    \put(0,8){$y_i$}

    \put(3,2){$\Delta t$}
    \put(3,1.8){\vector(-1,0){1}}
    \put(3,1.8){\vector(+1,0){2}}

    \put(6,1.7){$\Delta t_i$}
    \put(6,1.5){\vector(-1,0){4}}
    \put(6,1.5){\vector(+1,0){2}}
    
    \thinlines
    \put(2,2){\line(-1,0){1}}
    \put(2,2){\line(0,-1){1}}
    \put(8,8){\line(-1,0){7}}
    \put(8,8){\line(0,-1){7}}
    \put(5,7){\line(-1,0){4}}
    \put(5,7){\line(0,-1){6}}

  \end{picture}
  \caption{First order response prototype definition.  $\Delta t$ and
    $\Delta t_i$ are the times of the interpolation time and end point
    time relative to the prototype start.}
  \label{fig:definition}
\end{figure}

Our goal is to find a sequence of prototypes that fits the sequence of
events.  We wish to fit the entire data set, so the first and last
times are to coincide with the first and last times in the data set.
Therefore, given that we are fitting $N$ prototypes, we seek to find
$N-1$ transition times and $N$ parameter value sets.

A few key decisions ease optimisation.  Firstly, a linear term 
added to the exponential response ensures that the prototype
interpolates through the initial $(x_{i-1}, y_{i-1})$ and final
$(x_{i}, y_{i})$ points.  This does add any parameters to the
description.  The predicted value for the prototype at a given time
$t$ is shown in equation~\ref{eq:prototype}:
\begin{equation}
  \label{eq:prototype}
  y_p = y_{i-1} + y_i \left (1 - \underbrace{e^{\Delta
        t/\tau_i}}_{\textrm{exponential}} + \underbrace{\frac{\Delta
        t}{\Delta t_i}e^{\Delta t_i/\tau_i}}_{\textrm{linear}} \right)
\end{equation}

Secondly, the optimisation parameters were chosen to reduce coupling
in the problem parameters by using absolute times for each starting
point and constraining these times to be sequential rather than time
differences constrained to be positive.  This reduced the effect of
any one starting point on the error produced by the remaining fit
functions.

\subsection{Objective functions}
Two objectives were defined: the RMS error of the fit over all the
prototypes and the ``complexity'' of the fit, which was calculated as
\begin{equation}
  c = \sum_i^{N} \frac{1}{\tau_i}.
\end{equation}

This complexity measure works due to the addition of the linear
correction term, which dominates for large $\tau$, meaning that as
$\tau$ increases, one sees more of the linear behaviour and less of
the exponential.  Therefore, larger $c$ corresponds to greater
curvature of the fitting prototypes.

\subsection{Prototype to event mapping}
Each sequence of prototypes identified was mapped back to a sequence
of event types by using the following heuristics:
\begin{itemize}
\item If the difference between the start and end values is less
  than a cut-off value $\epsilon_c$, the prototype is taken to
  represent a constant event.
\item If the time constant is larger than a cut-off time constant
  $\tau_c$, it is taken as a ramp.
\item If neither of these holds, the prototype is a first order response.
\end{itemize}

The values of $\epsilon_c$ and $\tau_c$ are problem-dependant and
should be chosen to represent an insignificant change in $y$ and a
large time constant (in the chosen time units) respectively.




% Local Variables:
% TeX-master: "thesis"
% End:
