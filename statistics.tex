\chapter{Stationary and signal statistics}\label{chap:stats}
\begin{overview} 
  Deterministic models of chemical engineering systems assume that all the variables are known exactly.
  However, uncertainty typically exists in the model paramters, the inputs to the model or the measurements from the model.
  This chapter summarises the nomenclature used in the rest of the work in addition to covering the theory used to incorporate uncertainty into models.
\end{overview}

\section{Probability}
Simple random phenomena are commonly described using a model consisting of the following elements:\citep[1]{kulkarni1999modeling}
\begin{description}
\item[Sample space] Denoted $\Omega$, giving the set of all possible outcomes of the random phenomenon.  
  Particular outcomes are denoted $\omega$.
\item[Events] An event (typically denoted $E_i$) is a subset of the sample space.
\item[Event probabilities] Are written as $\prob{E}$ and are numbers between $0$ and $1$ representing the likelihood of occurrence of the event $E$.
\end{description}

Additionally, $\prob{E_2|E_1}$  represents the \emph{conditional probability} of event $E_2$ given that $E_1$ has indeed occurred.

It is clear that $\prob{\Omega}=1$, that is that there is a 100\% probability of something in the set of all possible outcomes occurs.
Also, $0 \leq \prob{E_i} \leq 1$ for any $E_i$ that is a subset of $\Omega$.

\section{Univariate random variables}
\label{sec:univ-rand-vari}
Random variables are typically defined as the functions that map the sample space of a random phenomenon to a real number.
Examples are the value of a particular temperature or pressure or the total number of heads shown after flipping a certain number of coins.

\subsection{Cumulative distribution function}
It is often useful to refer to the cumulative distribution function of a random variable, defined as 
\begin{equation}
  F(x) = \prob{X \leq x}, x \in (-\infty,\infty)
\end{equation}

Due to the fact that a random variable must take on a particular value in $(-\infty,\infty)$, $\lim_{x \to -\infty} F(x) = 0$ and $\lim_{x \to \infty} F(x) = 1$.
$F$ must also be nondecreasing ($x \leq y \implies F(x) \leq F(y)$) and right continuous ($\lim{\epsilon \to 0+} F(x+\epsilon) = F(x)$).

Random variables are called discrete if they have CDFs that contain discrete steps.

\subsection{Probability distribution function}
Random variables with for which a function $f$ exist such that 
\begin{equation}
  \label{eq:cdffrompdf}
  F(x) = \int_{-\infty}^xf(u)\dd u \quad \forall x \in (-\infty, \infty)
\end{equation}
are called continuous random variables.
If $F$ is continuous and piecewise differentiable, then $f$ completely determines $F$ and is called the probability distribution function (PDF) of the variable $X$.
Important properties of PDFs include $f(x) \geq 0 \quad \forall x \in \mathbb{R}$ (as no negative probability integrals are allowed) and $\int_{-\infty}^{\infty} f(u) \dd u = 1$ (due to the limit on $F$).  
It is however important to note there is no upper bound on $f$.

\section{Signal statistics}


\subsection{IEC 61131-3}
IEC 61131-3 is a standard that defines

\url{http://en.wikipedia.org/wiki/IEC_61131-3}

\subsection{Input identification}
Stochastic simulation of systems, especially those using Monte Carlo
methods, require good input scenarios to generate good output data.
It is common to make use of Markov processes to generate realistic
inputs based on historic data.  However, identification of ``events''
within historic data can be troublesome.  Much work has been done on
identification of events or trends in data (\citet{maurya.rengaswamy.ea2007fault}
give a good overview of trend analysis techniques).  Reducing process
signals to symbols representing qualitative event types rather than
quantitative data allows patterns to be found in events, or what
\citet{cheung.stephanopoulos1990representation} refer to as
episodes.  In this seminal work, the authors define a formal language
in terms of the 7 primitives shown in
figure~\ref{fig:stephanopoulosprimitives}.

\begin{figure}[htbp]
  \centering
  \setlength{\unitlength}{0.7em}
  {\small
  \begin{picture}(8,8)
    \put(0,0){\line(1,0){8}}
    \put(0,0){\line(0,1){8}}
    \put(7,1){D}
    \put(4,4){$(+,-)$}
    \thicklines
    \qbezier(1,1)(2,7)(7,7)
  \end{picture}
  }
  {\small
  \begin{picture}(8,8)
    \put(0,0){\line(1,0){8}}
    \put(0,0){\line(0,1){8}}
    \put(7,1){B}
    \put(1,4){$(+,+)$}
    \thicklines
    \qbezier(1,1)(7,2)(7,7)
  \end{picture}
  }
  {\small
  \begin{picture}(8,8)
    \put(0,0){\line(1,0){8}}
    \put(0,0){\line(0,1){8}}
    \thicklines
    \qbezier(1,1)(2,2)(7,7)
    \put(7,1){C}
    \put(1,5){$(+,0)$}
  \end{picture}
  }
  {\small
  \begin{picture}(8,8)
    \put(0,0){\line(1,0){8}}
    \put(0,0){\line(0,1){8}}
    \thicklines
    \qbezier(1,7)(6,7)(6.7,1)
    \put(7,1){G}
    \put(1,4){$(-,-)$}
  \end{picture}
  }
  {\small
  \begin{picture}(8,8)
    \put(0,0){\line(1,0){8}}
    \put(0,0){\line(0,1){8}}
    \thicklines
    \qbezier(1,7)(2,2)(6.7,1)
    \put(7,1){E}
    \put(4,4){$(-,+)$}
  \end{picture}
  }
  {\small
  \begin{picture}(8,8)
    \put(0,0){\line(1,0){8}}
    \put(0,0){\line(0,1){8}}
    \thicklines
    \qbezier(1,7)(4,4)(7,1)
    \put(7,1){F}
    \put(5,5){$(+,-)$}
  \end{picture}
  }
  {\small
  \begin{picture}(8,8)
    \put(0,0){\line(1,0){8}}
    \put(0,0){\line(0,1){8}}
    \thicklines
    \qbezier(1,4)(3,4)(7,4)
    \put(7,1){A}
    \put(4,6){$(0,0)$}
  \end{picture}
  }
  \caption[Episodic analysis primitives]{Episodic analysis primitives according to \citet{cheung.stephanopoulos1990representation}.}
  \label{fig:stephanopoulosprimitives}
\end{figure}

One problem with event-based approaches is that, to estimate the
likelihood of a state transition, at least one such a transition has
to be identified in the training data.  This means that such processes
are usually very data-intensive.  The same holds for episodic analysis
-- some patterns may go unnoticed because of misfitting.
Furthermore, it is difficult to determine an objective function for
fitting, as attempts to fit the data too accurately usually
lead to a loss of generality (the over-fitting problem,
described by \citet{arora.khot2003fitting} and others).

\section{Uncertainty}
The reason for stochastic simulation is the existence of uncertainty in
one or more aspects of the model.  If the model was perfect, a single
deterministic simulation would be a complete exploration of the
model.  In the steady state case, this means that solving the steady
state equations yield a single, reliable result.  

Uncertainty can be classified according its location as follows.

\subsection{Input uncertainty}
Input uncertainty refers to uncertainty about the values of the inputs
into the model.  In control problems and when processing natural
products, the properties of input streams may not be known in more
detail than expected ranges.

Input uncertainties are further subdivided into quantities that have
known distributions rather than fixed values and quantities that vary
over time, exhibiting known events.

\subsubsection{Input types}

\subsection{Parametric}
Parametric uncertainty is uncertainty in parameters of the model.  It
is usually assumed that the model shape is correct, but that
differences between simulated values and experimental data can be
explained by inaccurate parameter values.  Parametric uncertainty can
be due to incorrect models, which do not model variations accurately,
or to inaccurate measurements.  If model parameters change over time
in predictable ways, it is more proper to model these changes by
introducing more model equations than to mask them by assuming
parametric uncertainty.  The specific combination of parameters for a
simulation must therefore remain constant for that simulation.

A process for which the parameters remain constant in this way is
called an ergodic process
% http://en.wikipedia.org/wiki/Ergodic_process

It is possible to estimate the mean of an ergodic process by 
\begin{equation}
  \hat{\mu_T} = \frac{1}{2T} \int_{-T}^{T} x(t) \dd t
\end{equation}


\subsection{Model uncertainty}
Model uncertainty is the hardest to handle during simulations, as this
refers to uncertainty in the form of the model equations.  Model
uncertainty is different from parametric uncertainty as it implies
that no combination of parameters in the model can accurately capture
the behaviour of the system.  The boundary between model and
parametric uncertainty is often blurred by the fact that models are
often developed with additional parameters designed to make the model
more flexible.  This work does not address model uncertainty.

% Local Variables:
% TeX-master: "thesis"
% End:

