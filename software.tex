\chapter{Software}
\label{chap:software}
\begin{overview}
  Even though solving a set of model equations by hand can also be seen as simulation~\citehere, modern simulation will almost certainly involve using computers and computer programs.
  The act of simulation (ie the actual solution of the modelling equations) is also not the only process which benefits by using computers.
  Simulation generates data, which need to be stored and analysed in some way.
  The model descriptions themselves need to be encoded and stored.  
  This chapter reviews simulation-related software in this larger context, discussing current technology for developing models, solving the model
  equations and storing the resulting data.
  More attention is given to open-source software than to commercial software, as the framework is to use open source tools.
  The choices that have  been made for the modelling framework are interspersed in this discussion.
\end{overview}

\section{Model representations}

\subsection{Language based}

ProMOT \url{http://www.mpi-magdeburg.mpg.de/projects/promot}

Modelica

\subsection{XML based}

SBML \url{http://sbml.org/}


\subsection{Binary}


\section{Workflow}
NSF 

\subsection{What should workflow do?}

\subsection{Workflow tools}
% See:
% ~/Downloads/Documents/workflow
% ~/Downloads/Documents/workflow/scientific

\url{http://wiki.cogkit.org/index.php/Scientific_Workflow_Survey}

wftk \url{http://www.vivtek.com/wftk/perl_tutorial/08-workflow.html}
Perl-based single developer toolkit.  Good documentation about what
should go into a workflow tool.

Look into:
\begin{itemize}
\item \url{http://pegasus.isi.edu/wms}
\item \url{http://cppwfms.sourceforge.net}
\item \url{http://messagelab.monash.edu.au/Nimrod}
\end{itemize}

\subsection{Workflow specification languages}

\subsubsection{YAML}
\url{http://www.yawlfoundation.org} Very generic - examples are all
business workflows, has support for organisation model.  

\subsubsection{BPML}
Also for business.

\section{Concurrency}
\subsection{Local parallalism}

\section{Property packages}


\section{Simulators}
\subsection{Circuit simulators}
Spice
Gnu cap \url{http://www.gnu.org/software/gnucap/}
\subsection{Proprietary chemical systems}

\subsection{Modelica}
The Modelica project was spearheaded by \xxx, who started research on
modelling languages during his PhD studies in \xxx.  ``Modelica is a modern, strongly typed, declarative, and object-oriented language for modeling and simulation of complex systems.'' (\url{http://citeseerx.ist.psu.edu/viewdoc/summary?doi=10.1.1.139.7209})

\subsection{Stochastic simulation}
\subsubsection{Gnu MCsim}
From the MCSim users manual (\url{http://www.gnu.org/software/mcsim/mcsim.html}):
\begin{quote}
MCSim is a general purpose modeling and simulation program which can performs "standard" or "Markov chain" Monte Carlo simulations.
It allows you to specify a set of linear or nonlinear algebraic equations or ordinary differential equations. 
They are solved numerically using parameter values you choose or parameter values sampled from statistical distributions. 
Simulation outputs can be compared to experimental data for Bayesian parameter estimation (model calibration).
\end{quote}

\subsection{Proprietary simulators}
\subsubsection{gPROMS}

\subsubsection{SPEEDUP}

\subsubsection{HYSIS}

\subsubsection{CHEMCAD}

\subsection{Zero cost}
\subsubsection{Model.la}
% Model.la_Jerry_Bieszczad_phd.pdf
By a student of Stephanopolous. \url{http://web.mit.edu/modella/}

\subsubsection{COCO}
COCO simulator - Based on the Cape Open standard.  \url{http://cocosimulator.org/}

\subsection{Open source}

\subsubsection{ASCEND}
ASCEND from Carnegie Mellon university.  From the ASCEND website: \url{http://ascend.cheme.cmu.edu/}
\begin{quote}
  ASCEND is a system for solving systems of equations, aimed at engineers and scientists. It allows you to build up complex models as as systems constructed from simpler sub-models. Using ASCEND it is simple to play around with your model, examine its behaviour, and work out how it can best be solved. You can easily change which variables are fixed and which are to be solved, and you can examine the way in which the model is being solved.
\end{quote}
\subsubsection{EMSO}
%TODO: Check out EML the EMSO model library
EMSO process simulator from ALSOC (\url{http://www.enq.ufrgs.br/trac/alsoc}).

From the EMSO website:
\begin{quote}
EMSO is the acronym for Environment for Modeling, Simulation, and Optimization. 
EMSO is a graphical environment where the user can model complex processes simply selecting and connecting the equipment models. 

The main features of EMSO follows: 

\begin{itemize}
\item Entirely written in C++
\item A fairly portable code, currently
  available for Windows and Linux but can be compiled for other
  platforms if desired
\item It is an Equation-Oriented simulator The unique
  Equation-Oriented simulator with units-of-measurement checking for
  the equations
\item A large set of built-in functions
\item Models are written
  in a modeling language, the user does not need to be a programmer

\item Models are converted to system of equations in memory, no
  compilation or linking is needed
\item An open library of models, called
  EML
\item Built-in code for symbolic differentiation which enables the
  system to solve high-index problems
\item Built-in code for automatic
  differentiation which makes the system very efficient
\item Can make use
  of machine optimize BLAS routines
\item Currently support: 
  \begin{itemize}
  \item static simulation
  \item dynamic simulation
  \item static optimization
  \item parameter estimation of static models
  \item parameter estimation of
    dynamic models
  \end{itemize}
\item A
  graphical user interface which can be used to model development,
  simulation execution, and results visualizing
\item A system of PlugIns where the user can embed code written in C,
  C++ or FORTRAN into the models A very modular system - all solvers
  are DLL's and the user can even write their own NewSolver Models
\end{itemize}
\end{quote}

\subsubsection{DWSIM}
\label{sec:dwsim}

From the DWSIM website: \url{http://dwsim.inforside.com.br/}
\begin{quote}
DWSIM is a Chemical Process Simulator for Windows. Built on the Microsoft .NET 2.0 Platform and featuring a rich Graphical User Interface (GUI), DWSIM allows chemical engineering students and chemical engineers to better understand the behavior of their chemical systems by using rigorous thermodynamic and unit operations' models with no cost at all. Even better, they can see how the calculations are actually being done - DWSIM is open source, that is, its code is available to anyone who wishes to discover the "magic" behind it or just do some code browsing. 
\end{quote}

\subsubsection{OPSIM}
\label{sec:opsim}
Windows-only.  \url{http://sourceforge.net/projects/opsim/}

\section{Programming languages}
\citep{chaves.nehrbass.ea2006octave}

\section{Data formats}
\subsection{Time-series data}
Input and output time-series

\subsection{Properties}
See CAPE-Open

\subsection{Object data}
Data requirements of each unit/storage

\section{Data technologies}
\subsection{Relational databases}
\subsection{HDF5}

\section{CAPE-OPEN}
The CO-lan describes CAPE-OPEN as follows:
\begin{quote}
CAPE-OPEN standards are the uniform standards for interfacing process modelling software components developed specifically for the design and operation of chemical processes. 
They are based on universally recognized software technologies such as COM and CORBA. 
CAPE-OPEN standards are open, multiplatform, uniform and available free of charge. 
They are described in a formal documentation set.
\end{quote}

The documentation set comprises the following items:
\subsection{Thermodynamics and Physical Properties Interface Specification}

\subsection{Unit Operations Interface Specification}

\subsection{Chemical Reactions Interface Specification}

\subsection{Methods \& Tools Integrated Guidelines}

\subsection{Optimisation Interface Specification}

\subsection{Parameter Estimation and Data Reconciliation Interface Spec}
This document defines the interfaces required to do parameter estimation and data reconciliation.  
The parameter estimation problem is to find the parameters that correspond with the best match between the model and the data, while the data reconciliation problem is to find data that does not correspond to the model.  
Both of these activities require a model of the system or the ability to run the model and data from the system.  

\subsection{Partial Differential Algebraic Equations Interface Specification}

\subsection{Petroleum Fractions Interface Specification}

\subsection{Physical Properties Data Bases Interface Specification}

\subsection{Planning and Scheduling Interface Specification}

\subsection{Simulation Context COSE Interface}


\subsection{Identification Common Interface}
\subsection{Parameter Common Interface}
\subsection{Collection Common Interface}
\subsection{Error Common Interface}
\subsection{Persistence Common Interface}
\subsection{Utilities Common Interface}

\section{Complex event processing}
% TODO: Flesh out
Identification of multiple events in a stream of events as a single event is being done under the name ``Complex event processing'' \url{http://en.wikipedia.org/wiki/Complex_event_processing}.  
This is mostly done in the field of business intelligence.  

% Local Variables:
% TeX-master: "thesis"
% End:

