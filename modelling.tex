\chapter{Modelling and Simulation}

\begin{overview}
Modelling and simulation are two sides of the same activity.
This chapter defines the terminology used in dynamic modelling and covers the techniques of simulation.
The focus here is on understanding different approaches to modelling and simulation with a view to fault finding.
\end{overview}

\section{Definitions}
The terminology used in modelling and simulation are less rigourously defined than may be desirable.
This means that a particular word may have several meanings depending on which source is being consulted.
\citet[4--10]{cellier1991continuous} uses the following definitions:  
\begin{description}
\item[System] That which is distinguished as a system.  
This refers to the idea that anything can be a system, as long as the boundaries are well-defined.
\item[Inputs] Variables from outside the system (the environment of the system) that affect it.
\item[Outputs] Variables from inside the system that can be observed.
\item[Experiment] Extracting data from a system by exercising it through its inputs
\item[Model] Anything to which an experiment can be applied in order to answer questions about a system.
\item[Simulation] An experiment performed on a model (as opposed to the ``real'' system).
\end{description}

The word ``observed'' in the definition of the outputs refers to being able to determine the value.  
In the case of mathematical models, all the variables calculated during simulation are observable in this sense.

\section{Modelling}
\subsection{Types of model}
\citet{karplus1977spectrum} identified that mathematical models cover a spectrum of ranging from ``black box'' to ``white box'' models, covering several shades of grey in between as shown in figure~\ref{fig:karplusspectrum}.
\begin{figure}[htbp]
  \centering
  \includegraphics[width=\halfwidth]{karplus_spectrum}
%  \input{graph/karplus_spectrum.pdf_tex}
  \caption[Modelling spectrum]{Modelling spectrum (adapted from~\citet{karplus1977spectrum})}
  \label{fig:karplusspectrum}
\end{figure}
\citet[15]{cellier1991continuous} is of the opinion that most chemical engineering problems are toward the white end of the spectrum.
For this reason, this work will focus on continuous-time models.



\subsection{Continuous-time models}
\label{sec:cont-time-models}
Engineering continuous-time models are mostly one of the following types.
Distributed parameter models are described by partial differential equations (PDEs), while lumped parameter models are described by ordinary differential equations (ODEs).
Although distributed parameter models are widely used in, for instance, chemical reactor modelling, most dynamic flowsheeting is done via lumped parameter models.

Ordinary differential equations (ODEs) are described in general by the following form:
\begin{equation}
  \label{eq:odeform}
  f(x, \dot{x}, \ddot{x}, \ldots, x^{(n)}, t) = 0
\end{equation}
Here, $x$ is often described as the state of the system, $\dot{x}$ is the state derivative (and $\ddot{x}$ is the second derivative, etc) and $t$ is most commonly time.
The highest derivative is called the order of the ODE.
If this equation can be manipulated such that the highest state derivative can be written independently, as
\begin{equation}
  \label{eq:odeform_explicit}
  \dot{x} = g(x, t)
\end{equation}
it is described as an explicit ODE.  
If it cannot, it is called implicit.  
If, additionally, $g$ can be decomposed into two functions such that 
\begin{equation}
  \label{eq:odeform_seperable}
  \dot{x} = g(x, t) = h(x)y(t)
\end{equation}
it is described as a separable ODE.

The above equations have been written with scalar arguments.  A system of such equations can be represented compactly by expressing the states as a state vector $\vect{x}$.  
This also enables the convenient representation of higher order derivatives by introducing dummy variables equal to the derivatives.  
Given for instance a second order ODE $f(x, \dot{x}, \ddot{x}, t)=0$, we could define a new variable $y=\dot{x}$, then use a state vector $\vect{z}=[x~y]^\transpose$ so that $\dot{\vect{z}} = [\dot{x}~\dot{y}]^\transpose = [\dot{x}~\ddot{x}]^\transpose$, and therefore $f(\vect{z}, \dot{\vect{z}}, t) = 0$.  

The introduction of systems of equations in the form of equation~\ref{eq:odeform} also leads to the possibility that some of these equations are purely algebraic.
Such equations are known as differential-algebraic equations.

% see http://en.wikipedia.org/wiki/Differential_algebraic_equation
In this case, the model equations can be written as
\begin{equation}
  \label{eq:daeform}
  F(\dot{\vect{x}}, \vect{x}, \vect{y}, t) = 0
\end{equation}
to make the distinction between $\vect{x}$, variables which take part in the differential part of the equations and $\vect{y}$, the variables which only take part in the algebraic equations.

The formulation in equation~\ref{eq:daeform} is known as fully implicit.  
If it is possible to separate the equations into a differential part and an algebraic part a semi-implicit form is obtained:
\begin{align}
  \label{eq:daesemiimplicit}
  f(\dot{\vect{x}}, \vect{x}, \vect{y}, t) &= 0 \\
  g(\vect{x}, \vect{y}, t) &= 0
\end{align}

% TODO: http://www.scholarpedia.org/article/Differential-algebraic_equations#Index_and_mathematical_structure
Of particular interest is the concept of the differentiation index\index{index} of the DAE.
The index may be loosely defined as the number of times the DAE has to be differentiated to $\vect{x}$ before yielding an ODE.
By this definition, ODEs have index 0.

% TODO: http://en.wikipedia.org/wiki/Pantelides_algorithm
%  Pantelides, The Consistent Initialization of Differential-Algebraic Systems, SIAM J. Sci. and Stat. Comput. Volume 9, Issue 2, pp. 213–231 (March 1988) the original paper where the algorithm is described
% TODO: find good reference for Pantalides algorithm
% TODO: Read http://www.ece.arizona.edu/~cellier/ece449_lecture.html
Many simulation packages have the ability to reduce\index{index!reduction} the index of a DAE automatically using Pantalides's algorithm 

For the purposes of modelling real systems, these equations will all be modified by taking into account the effect of the inputs to the system.
This is often indicated by the vector $\vect{u(t)}$.

% TODO: Mention partial differential equations

\subsection{Discrete time models}
The differential equations discussed in section~\ref{sec:cont-time-models} assume or imply that time proceeds in a continuous fashion.
If a process is discontinuous or batch-like, this may not be an accurate description.
In such cases, a model may be described in discrete form, where the model essentially describes the behaviour of the system at distinct points in time.
These models also take one of several forms, discussed below:
 
\subsubsection{Difference equations}
The evolution of a system state is clearly a function of the inputs into the system and the state itself.
It is convenient if the relationships are linear, so that one may write
\begin{equation}
  \label{eq:diff}
  x_t = \mu + \sum_{i=1}^{p}\phi_i x_{t-i} + \sum_{i=0}^{q}\theta_i u_{t-i}
\end{equation}
Such difference equations are frequently encountered in econometrics and time series analysis, where specific versions are distinguished.
simulation if the current state can be calculated from previous states here the future state of the model is described in terms of its past states.

\subsection{Classes}
AR Auto-regressive % http://en.wikipedia.org/wiki/Autoregressive_model
MA Moving average % http://en.wikipedia.org/wiki/Moving_average_model
I Integrative
GARCH
PARCH

\section{Simulation}
The solution of modelling equations is called simulation.  
Analytic solutions for some ODEs exist, while several algorithms exist for the numeric solution of ODEs, DAEs and PDEs.

% TODO: Flesh this out!

\section{Software tools}
\subsection{Modelling environments}
The modern trend in software tools is to support the engineer in modelling systems by providing modules that can easily be connected to one another to represent the system at hand.
Typically, such systems will provide mechanisms to prevent logical errors in the connection as well as simulation and validation tools.
% TODO: Merge with other discussion of simulation tools.

\section{Uncertainty}
The reason for stochastic simulation is the existence of uncertainty in one or more aspects of the model.  
If the model was perfect, a single deterministic simulation would be a complete exploration of the model.  
In the steady state case, this means that solving the steady state equations yield a single, reliable result.  

Uncertainty can be classified according to its location as follows.

\subsection{Input uncertainty}
Input uncertainty refers to uncertainty about the values of the inputs into the model.  
In control problems and when processing natural products, the properties of input streams may not be known in more detail than expected ranges.

Input uncertainties are further subdivided into quantities that have known distributions rather than fixed values and quantities that vary over time, exhibiting known events.

%\subsubsection{Input types}

\subsection{Parametric}
Parametric uncertainty is uncertainty in parameters of the model.  
It is usually assumed that the model shape is correct, but that differences between simulated values and experimental data can be explained by inaccurate parameter values.  
Parametric uncertainty can be due to incorrect models, which do not model variations accurately, or to inaccurate measurements.  
If model parameters change over time in predictable ways, it is more proper to model these changes by introducing more model equations than to mask them by assuming parametric uncertainty.  
The specific combination of parameters for a simulation must therefore remain constant for that simulation.

\subsection{Model uncertainty}
Model uncertainty is the hardest to handle during simulations, as this refers to uncertainty in the form of the model equations.  
Model uncertainty is different from parametric uncertainty as it implies that no combination of parameters in the model can accurately capture the behaviour of the system.  
The boundary between model and parametric uncertainty is often blurred by the fact that models are often developed with additional parameters designed to make the model more flexible.  
This work does not address model uncertainty.

% Local Variables:
% TeX-master: "thesis"
% End:
