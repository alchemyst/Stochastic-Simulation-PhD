\chapter*{Synopsis}\addcontentsline{toc}{section}{Synopsis}

Chemical engineering process modelling and simulation pose significant challenges to the computer program developer.
Chemical processes are invariably described by non-linear equations such as chemical reaction kinetics, flow-pressure relationships and physical properties.
Dynamic simulation of such systems involves the solution of sets of non-linear differential and algebraic equations.
There is also uncertainty associated with the model equations themselves (model uncertainty), their parameters (parametric uncertainty) and the inputs into the model (input uncertainty).
Tools that aid chemical engineers in the solution of these problems have been successfully commercialised and enjoy a measure of success, although the adoption of dynamic and stochastic simulation packages is lagging behind the steady-state flowsheeting tools.

Commercial software solutions can be prohibitively expensive in addition to confining users to adhere to proprietary standards.
 The Free Software and Open Source movements have made inroads into providing non-proprietary alternatives to many commercial software packages, which has encouraged the adoption of open standards.
 Pressure from the commercial users of simulation software also led to the development of the CAPE-Open standards to ensure interoperability between proprietary platforms.

This work presents a framework for the development of stochastic dynamic simulations of chemical processes using only free and open source software.
 The large problem of stochastic dynamic simulation has been broken down into stages:
\begin{enumerate}
\item Input modelling using Markov chain models trained on process data or seeded by hand in addition to stationary distribution models. 
  This enables dynamic scenarios to be handled with the minimum of special case code generation.
\item Process modelling using an object-oriented approach in the Modelica language. 
  Modelica has an open source implementation called OpenModelica and is an open standard modelling language.
\item Monte Carlo simulation using extensions to the OpenModelica compiler that ease parallel simulations
\item Postprocessing, including visualisation and statistical analysis. 
  Statistics that are generated can be used for control or validation purposes.
\end{enumerate}

\noindent \textbf{KEYWORDS: simulation, chemical processes, modelica, open source, stochastic} 

\begin{otherlanguage}{afrikaans}
\chapter*{Sinopsis}\addcontentsline{toc}{section}{Sinopsis}
Chemiese ingenieurswese prosesmodellering en -simulasie behou groot uitdagings vir die rekenaarprogramontwikkelaar. 
Chemiese prosesse word dikwels beskryf deur nie-lini\^ere vergelykings soos chemiese reaksiekinetika, vloei-druk verhoudings en fisiese eienskappe. 
Dinamiese simulasie van sodanige stelsels behels die oplossing van stelle van nie-line\^ere differensiaal- en algebra\"iese vergelykings. 
Daar is ook onsekerheid in verband met die model vergelykings hulself (model onsekerheid), hul parameters (parametriese onsekerheid) en die insette tot die model (inset-onsekerheid). 
Verskeie programmatuur wat chemiese ingenieurs steun in die oplossing van hierdie probleme is suksesvol gekommersialiseer en geniet 'n mate van sukses, hoewel die opname van dinamiese en stogastiese simulasie pakkette stadiger is as di\'e van gestadigde toestand pakette.

Kommersi\"ele sagteware oplossings is ook dikwils ontoereikend duur en mag gebruikers beperk tot hulle eie standaarde. 
Die Vrye Sagteware en Oopbronkode bewegings maak vordering in die verskaffing van oop alternatiewe vir baie kommersi\"ele sagteware pakkette, wat ook oop standaarde aanmoedig.
Druk van die kommersi\"ele gebruikers van simulasie sagteware het ook gelei tot die ontwikkeling van die ``Cape-Open'' standaarde om onderlinge werkbaarheid te verseker tussen die verskeie platforms. 

Hierdie werk bied 'n raamwerk vir die ontwikkeling van stogastiese dinamiese simulasies van chemiese stelsels deur van slegs gratis en oopbronsagteware gebruik te maak. 
Die groot probleem van dinamiese stogastiese simulasie is afgebreek in fases:
\begin{enumerate}
\item Inset modellering deur gebruik te maak van Markovketting-modelle wat identifiseer is uit prosesdata of handgekeurde datastelle saam met stasion\^ere verdeling modelle.
  Dit maak dit moontlik om dinamiese scenario's te hanteer met die minimum van kode vir spesiale gevalle.
\item Prosesmodellering met behulp van 'n objek-geori\"enteerde benadering in die Modelica taal.
  Modelica het 'n oop bron implementering genaamd OpenModelica en is 'n oop standaard modelleringstaal.
\item Monte Carlo-simulasie met uitbreidings aan die OpenModelica samesteller om parallel simulasies te vergemaklik.
\item Nabewerking, insluitend visualisering en statistiese analise.
  Statistieke wat so gegenereer is kan gebruik word vir beheer of validering.
\end{enumerate}

\bigskip 

\noindent \textbf{KERNWOORDE:} simulasie, chemiese prosesse, stogasties

\end{otherlanguage} 

% Local Variables: 
% TeX-master: "thesis" 
% End:

